\section{Genetic Algorithm}

Our implementation follows the basic schema of every GA\cite{Thede2004}. Creation and evaluation of an initial
population ($p$), and improvement of that population until a termination condition is met. The population
refinement  is done via a \emph{selection} of two or more parents, generation of new childs from the
\emph{crossover} of these parents, application of a \emph{mutation} to diversify the population and a
evaluation of the childs to select and \emph{refine} the new population that will be used in the next
iteration of the loop.

In order to improve the performance of the algorithm some domain-specific components were implemented,
mainly in the \nameref{ss:crossover} (\ref{ss:crossover}) and \nameref{ss:mutation} (\ref{ss:mutation})
steps. This section will offer a brief description of the implementation of every facet of the GA.

\subsection{Chromosome encoding\label{ss:chromosome}}

The chromosome encoding that we used in our individuals is an array $S$ of size $n$ where each index would
point to the randomly assigned hub of each node. On top of that, these individuals are built with the
necessary contracts to ensure that they always contain $p$ unique values stored in the chromosome, that
way we are ensuring that every chromosome is a valid solution with $p$ hubs.

\begin{equation}
  S = [1, 1, 5, 1, 5, 5, 9, 1, 5, 9 ],\quad p = 3 \label{eq:chromosome_example}
\end{equation}

An example of this chromosome to a problem of size $n = 10$ and $p = 3$ is the one shown at (\ref{eq:chromosome_example}).
In this example the node 1 is assigned to itself as indicated by $S_{1}=1$, the node 2 is allocated also
to node 1, the node 3 to another hub placed in 5 and so on. See also that $U(S)=\{1,5,9\}$ fullfilling
the $|U(S)|=3$ requirement to have $p=3$ hubs.

The generation of the initial population of individuals is random, selecting three different values from
${1,\dots,n}$ and randomly assigning them to the different hubs. To ensure that all three have at least
one assigned hub, these first random chromosome will always allocate each hub to itself (\ref{eq:autoalloc}).

\begin{equation}
  S_{i}=i, \forall i \in U(S) \label{eq:autoalloc}
\end{equation}

\subsection{Selection}

In the selection step the strategy used is the \emph{Binary Tournament}. We pick two random individuals from
the population and select the one with the higher fitness. In case of equal fitness the first candidate
gets the priority. Both random individuals can be the same, in that case, that individual is ensured to
be picked for the crossover step.

This is done twice to select a total of two parents.  Both selected parents can also be the same individual,
in that case the child generated will be a clone of the parent, but can still provide a better solution with
the mutation.

\subsection{Crossover\label{ss:crossover}}

The crossover applied in the algorithm is based on the \emph{Single Point Crossover} strategy, but adding
two possible modifications to ensure the validity of the solutions, as the merged chromosome can
have between $1$ and $2p$ nodes, as it is shown in (\ref{eq:bad_chromosome}) with two parents and a
cutting point that can generate two invalid chromosomes, the first one with a number of hubs below
$p$ and the second one with almost twice the value of $p$.

\begin{equation}
  \label{eq:bad_chromosome}
  [1,1,1,1,2,3] x [5,1,5,4,1,1] \xRightarrow{x=4} [1,1,1,1,1,1], [5,1,5,4,2,3]
\end{equation}

This single point version generates a cutting point $x \in \{1,\dots,n\}$, creates a new child, and
then replicates the hub of every node $i$ from the left parent while $i<=x$ and the right
parent for every $x<i<=n$. If $x=n$ then the child will be an exact replica of the left parent.

Then it evaluates the resulting child looking at its number of hubs. If it's exactly $p$ the child
is deemed valid and returned. If it's below $p$, new hubs are generated until $p$ is satisfied, reallocating
each of these nodes converted to hubs to themselves. If it's above $p$, random hubs are selected and
removed until the child matches $p$, every orphaned node from this purge is rellocated to one of the
remaining nodes. See the algorithm \ref{alg:crossover} for the full breakdown.

\begin{algorithm}[H]
  \caption{Crossover}
  \label{alg:crossover}
  \begin{algorithmic}[1]
    \Require{$L$ (left parent)}
    \Require{$R$ (right parent)}
    \Require{$p$}
    \Ensure{valid child}
    \Function{RemoveHub}{$S,p$}
      \While{$length(S.hubs)>p$}
        \State $x \gets randomNode(S)$
        \State $y \gets randomNode(S - nodeToRemove)$
        \For{$n$ in $0, length(S)$}
          \If {$S_{n} = x$}
            \State $S_{n} \gets y$
          \EndIf
        \EndFor
      \EndWhile
    \EndFunction
    \item[]
    \Function{AddHub}{$S,p$}
      \While{$length(S.hubs)<p$}
        \State $x \gets randomNode(S - S.hubs)$
        \State $S_{x} \gets x$
      \EndWhile
    \EndFunction
  \item[]
  \State $x \gets randomPoint(N)$
  \State $C \gets L.replica()$
  \For{$i$ in ($x$, $length(child)$)}
    \State $C_{i} \gets R_{i}$
  \EndFor
  \If{$length(C.hubs) > p$}
    $RemoveHub(C,p)$
  \ElsIf{$length(C.hubs) < p$}
    $AddHub(C,p)$
  \EndIf
  \State \Return C
  \end{algorithmic}
\end{algorithm}

\subsection{Mutation\label{ss:mutation}}

The mutation applied to the chromosome is the a hub reassignment. A random non-hub node is
picked and transformed into a hub. Every sibling node assigned to the same hub is reallocated
to this new hub, including the previous hub (see Alg. \ref{alg:mutation}).
The probability to perform this mutation can be specified before running the algorithm,
and different values ($\{0.05., 0.1, 0.25, 0.5\}$) were tested in the following section 
(\ref{s:computational_results}) to find which probability performed the best.

\begin{algorithm}[H]
  \caption{Mutation}
  \label{alg:mutation}
  \begin{algorithmic}[1]
    \Require{$S$ (chromosome)}
    \State $h \gets randomNode(S - S.hubs)$
    \For{$n$ in $0, length(S)$}
    \If {$S_{n} = S_{h}$}
        \State $S_{n} \gets h$
      \EndIf
    \EndFor
  \end{algorithmic}
\end{algorithm}

\subsection{Replacement}

To perform the replacement that generates the population of the next interaction a simple
elitist approach was followed. Adding the new child to the current population in the place
of the current worst individual if it has a better fitness.


